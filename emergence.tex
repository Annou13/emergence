\documentclass[grid,avery5371]{flashcards}

% include font icons
\usepackage{fontawesome}

% format url
\usepackage{hyperref}

% specify the variant (AV, AI, LLM, etc) among available card decks.
\newcommand{\deck}{floss}

% read card info from external file
\usepackage{datatool}
\DTLloaddb{cards}{tools.csv}

% packages d'encodage
\usepackage[utf8]{inputenc}
\usepackage[T1]{fontenc}
\usepackage{ebgaramond}

\usepackage{multicol}

\geometry{headheight=12pt,footskip=4pt}
\usepackage{fancyhdr}
\pagestyle{fancy}
\fancyhf{}
\renewcommand{\headrulewidth}{0pt}
\chead{\small ({\sc \deck} deck)}

\title{\game}
\author{Olivia}
\cardbackstyle[\large]{plain}
\cardfrontstyle[\large]{headings}

\begin{document}

\cardfrontfoot{\faicon{creative-commons}}

% Loop through the CSV and create flashcards
\DTLforeach{cards}{%
    \Content=content,%
    \Definition=définition%
}{%
    % Create card front with the tag content
    \begin{flashcard}[front]{\Content}
    \end{flashcard}
    
    % Create card back with the definition
    \begin{flashcard}[back]{\Definition}
    \end{flashcard}
}

\end{document}
