% adopted from "Riddles in the Dark" template on Overleaf
% incorporated ideas from "Mailmerged Conference Name Cards" template 
\documentclass[grid,avery5371]{flashcards}

% include font icons
\usepackage{fontawesome}

% format url
\usepackage{hyperref}

% specify the variant (AV, AI, LLM, etc) among available card decks.
\newcommand{\deck}{floss}
\newcommand{\game}{emergence}

% read card info from external file
\usepackage{datatool}
%% The "database" is a comma-separated values (CSV) file.
%% The first line should contain the column headers, without space characters, e.g.
%% Name,JobTitle,Department
%%
%% If a field value contains a comma, then the field value needs to be surrounded with double quotes, e.g. 
%% John Smith,Lecturer,"School of Science, Mathematics and Engineering"
%%
%% Spreadsheet applications can usually export such a .csv file.
%%
%% If field values are expected to contain LaTeX special characters like $, &, then use \DTLloadrawdb{data}.csv instead.
\DTLloaddb{cards}{cards-\deck.csv}


\usepackage[utf8]{inputenc}
\usepackage[T1]{fontenc}
\usepackage{ebgaramond}

\usepackage{multicol}

\geometry{headheight=12pt,footskip=4pt}
\usepackage{fancyhdr}
\pagestyle{fancy}
\fancyhf{}
\renewcommand{\headrulewidth}{0pt}
\chead{\small \game ({\sc \deck} deck) }

\title{\game}
% compiled by:
\author{Matthijs den Besten}
% see resitory at https://github.com/mdbesten/emergence
\cardbackstyle[\Huge]{plain}
\cardfrontstyle[\large]{headings}



\begin{document}


\cardfrontfoot{\game\qquad\faicon{creative-commons}}

%\begin{flashcard}[\faicon{gamepad}\quad Gameplay Instructions]{
%\begin{turn}{180}
%\Rotatebox[origin=c]{90}{%
%{\bf\small During the game}
%\begin{multicols}{2}
%\begin{enumerate}\tiny \setlength{\itemsep}{.5ex}
%    \item Take notes on ideas that you have not thought about before
%    \item Some of the cards are causes, others are effects. Don’t worry about what the game designers intended with each card, go where the discussion is best.
%    \item Some of the cards may be upsetting. (Such as, a person is abused.)
%    \item It’s fine to take time to have discussion.
%    \item Try not to get sidetracked, though!
%\end{enumerate}
%\end{multicols}
%}
%\end{turn}{180}
}
%{\bf\small \game---Gameplay Instructions}
%\begin{multicols}{2}
 %   \begin{enumerate}\tiny \setlength\itemsep{.2ex}
  %      \item All players draw 5 event cards (\faicon{reply}) from their stack
   %     \item Throw the dice to choose the first player who will be the Card Czar.
    %    \item The Card Czar then pulls a prompt card (\faicon{tasks}) and reads it to the group
     %   \item All other players then put 1 event card face down on the table.   
      %  \item The Card Czar then flips and reads each white card out loud.
      %  \item The Card Czar then picks one of the event cards to further discuss. +1 point goes to the player whose card was chosen.
      %  \item The group then discusses further what else could go wrong based on the chosen card. People can award +1 point anyone who makes a good point in discussion.
      %  \item After the discussion dissipates after a few minutes, another player becomes the Card Czar based on the throw of a dice. Each player then draws a new white card, so that they again have 5 cards in their hand.
    %\end{enumerate}
    %\end{multicols}
%\end{flashcard}

% Adopted from http://www.unm.edu/~unmvclib/gamification/assessment/gameevaluationcriteria.pdf
\begin{flashcard}[\faicon{meh-o}\quad Game Evaluation Criteria]{
    \begin{itemize}\tiny \setlength{\itemsep}{.1ex}
        \item Category Biases
        \item Scale 1 to 7
        \begin{multicols}{3}
        \begin{itemize}
            \item Clarity
            \item Flow
            \item Balance
            \item Length
            \item Integrity
            \item Fun
        \end{itemize}
        \end{multicols}
        \begin{multicols}{2}
        \item Strongest Point
        \item Weakest Point
        \item One Change
        \item Comparable Games
        \end{multicols}
    \end{itemize}
    }
    {\bf\small \game---Game Evaluation}
    \small
    \begin{tabular}{l|c|c|c|c|}
        & 1 & 3 & 5 & 7\\
    \hline
    {\em Clarity} & Opaque & Muddy & Transparent & Water-clear \\
    {\em Flow} & Cumbrous & Fraught & Smooth & Natural \\
    {\em Balance} & Broken & Fluky & Sensible & Fair\\
    {\em Length} & Wrong & Unfit & Apt & Perfect \\
    {\em Integrity} & Eristic & Erratic & Coherent & Sound \\
    {\em Fun} & Offputting & Boring & Engaging & Exciting \\
    \hline
    \end{tabular}
\end{flashcard}


\DTLforeach{cards}{%
    %% Map each column header in your .csv file to a command
	\Description=description,%
    \Category=family,%
    \Id=period.id,%
    \Tool=discovery,
    \Dependson=dependson%
}{%%%  Start designing your output text!
\begin{flashcard}[\deck:\quad\Id\quad\DTLifnull{\Tool}{}{\Tool}]{%
%\begin{verse}[\versewidth]
\Description

Reqs: \Dependson
%\end{verse}
}
\faicon{tasks}\quad{\sc \Tool}\\
\end{flashcard}
}




\end{document}
